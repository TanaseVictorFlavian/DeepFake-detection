\begin{abstractpage}

\begin{abstract}{english}

The evolution of the artificial intelligence models in recent years has reinvented the creation of content on social media. In the same time, the progress in Computer Vision has been viewed by others as an opportunity to spread misinformation and create malicious content that can be hardly detected by the naked eye.

What has started as academic research by Justus Thies et al. \cite{thies2016face2face} pioneered a new way to fabricate content, capable to manipulate images and videos with unprecedented realism. This type of content can be used in many harmful ways such as blackmailing, political manipulation, fraudulent schemes, identity theft, and the creation of explicit material. Often, public figures are the primary targets of such attacks.

In order to fight disinformation, this work aims to provide the public with an artificial intelligence model with a web interface, that can be easily used to detect or at least raise suspicion over possibly fabricated images or videos. 
\end{abstract}

\end{abstractpage}