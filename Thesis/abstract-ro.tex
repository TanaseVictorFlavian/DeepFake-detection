\begin{abstractpage}

\begin{abstract}{romanian}

Evoluția capabilităților algoritmilor de inteligență artificială din ultimii ani a reinventat crearea de conținut în sfera digitală. În același timp, progresul în câmpul vederii artificiale a fost văzut de către unele entități ca o oportunitate de a răspândi dezinformare sau de a crea conținut malițios greu de detectat cu ochiul liber. 

Ceea ce a început ca cercetare academică realizată de către Justus Thies et al. \cite{thies2016face2face} a dat naștere unui mijloc de fabricare de conținut contrafăcut, capabil să manipuleze imagini și videoclipuri cu un realism nemaivăzut. Acest tip de conținut poate fi folosit in diverse scopuri cu rea-voință precum: șantaj, manipulare politică, înșelătorie, furt de identitate sau creare de conținut neadecvat, de cele mai multe ori aceste atacuri vizând persoane publice. 

În scopul combaterii dezinformării, această lucrare are ca obiectiv punerea la dispoziție către public a unui model de inteligență artificială cu o interfață web, ce poate fi ușor de utilizat pentru detectarea sau cel puțin ridicarea suspiciunii asupra imaginilor sau videoclipurilor posibil fabricate.

\end{abstract}
    
\end{abstractpage}